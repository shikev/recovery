%You can start at the line started with "\solutions{"
\documentclass[tikz]{article}
\usepackage{url,amsfonts, amsmath, amssymb, amsthm,color, enumerate,hyperref}

% Page layout
\setlength{\textheight}{8.75in}
\setlength{\columnsep}{2.0pc}
\setlength{\textwidth}{6.5in}
\setlength{\topmargin}{0in}
\setlength{\headheight}{0.0in}
\setlength{\headsep}{0.0in}
\setlength{\oddsidemargin}{0in}
\setlength{\evensidemargin}{0in}
\setlength{\parindent}{1pc}
\newcommand{\shortbar}{\begin{center}\rule{5ex}{0.1pt}\end{center}}
%\renewcommand{\baselinestretch}{1.1}
% Macros for course info
\newcommand{\courseNumber}{EECS 477}
\newcommand{\courseTitle}{Introduction to Algorithms}
\newcommand{\semester}{Fall  2015}
% Theorem-like structures are numbered within SECTION units
\theoremstyle{plain}
\newtheorem{theorem}{Theorem}[section]
\newtheorem{lemma}[theorem]{Lemma}
\newtheorem{corollary}[theorem]{Corollary}
\newtheorem{proposition}[theorem]{Proposition}
\newtheorem{statement}[theorem]{Statement}
\newtheorem{conjecture}[theorem]{Conjecture}
\newtheorem{fact}{Fact}
%definition style
\theoremstyle{definition}
\newtheorem{definition}[theorem]{Definition}
\newtheorem{example}{Example}
\newtheorem{problem}[theorem]{Problem}
\newtheorem{exercise}{Exercise}
\newtheorem{algorithm}{Algorithm}
%remark style
\theoremstyle{remark}
\newtheorem{remark}[theorem]{Remark}
\newtheorem{reduction}[theorem]{Reduction}
%\newtheorem{question}[theorem]{Question}
\newtheorem{question}{Question}
%\newtheorem{claim}[theorem]{Claim}
%
% Proof-making commands and environments
\newcommand{\beginproof}{\medskip\noindent{\bf Proof.~}}
\newcommand{\beginproofof}[1]{\medskip\noindent{\bf Proof of #1.~}}
\newcommand{\finishproof}{\hspace{0.2ex}\rule{1ex}{1ex}}
\newenvironment{solution}[1]{\medskip\noindent{\bf Problem #1.~}}{\shortbar}

%====header======
\newcommand{\solutions}[5]{
%\renewcommand{\thetheorem}{{#2}.\arabic{theorem}}
\vspace{-2ex}
\begin{center}
{\courseNumber, \courseTitle
\hfill {\bf\Large {#1}}\\
\semester, University of Michigan, Ann Arbor \hfill
{\em Date: #4}}\\
\vspace{-1ex}
\hrulefill\\
\vspace{4ex}
{\LARGE Homework {#3} Solutions}\\
\vspace{2ex}
\end{center}
\begin{itemize}
\item {\bf \Large Group Members: {#2}}
\item {\bf \Large Collaborators: {#5}}
\end{itemize}
\noindent
\shortbar
\vspace{3ex}
}
% math macros
\newcommand{\defeq}{\stackrel{\textrm{def}}{=}}
\newcommand{\Prob}{\textrm{Prob}}
%==
\begin{document}
%%%%%%%%%%%%%%%%%%%%%%%%%%%%%%%%%%%%%%%%%%%%%%%%%
%\solutions{Group_Name}{list of student_name (unique_name)}{Problem Set Number}{Date of preparation}{Collaborators}
\solutions{study group}{Xinyan Zhao (zhaoxy), Kevin Shi (shikev), Quincy Davenport(quincyd)}{1}{\today}{Xinyan, Kevin, Quincy}
%%%%%%%%%%%%%%%%%%%%%%%%%%%%%%%%%%%%%%%%%%%%%%%%%
%\renewcommand{\theproblem}{\arabic{problem}} 
%%%%%%%%%%%%%%%%%%%%%%%%%%%%%%%%%%%%%%%%%%%%%%%%%
%
% Begin the solution for each problem by
% \begin{solution}{Problem Number} and ends it with \end{solution}
%
% the solution for Problem 1
\begin{solution}{1}\\\\
The main idea is to create a recurrence relation that describes the algorithm and figure out asymptoptic behavior.\\
We will first describe the recurrence relation for the algorithm. Then we will prove that the runtime of this algorithm is loglinear.\\
Recurrence relation:\\\\
$T(n) = T(\frac{n}{4}) + T(\frac{3n}{4}) + \theta(n)$\\\\
where $T(\frac{n}{4})$ is the time cost of finding the median of medians\\\\
where $T(\frac{3n}{4})$ is the time cost of the recurrence in which we reduce the problem size to $\frac{3n}{4}$ (we can
eliminate a quarter of all values every iteration of the algorithm).\\\\
and $\theta(n)$ is the time to find the median of every group of four.\\
\\
Next we will calculate the upper bound of this algorithm:\\
\\
We know $T(x) = x^{\rho}(1 + \int_{1}^{x} \frac{g(u)}{u^{\rho + 1}}du) \, ,where \, \rho = \sum_{i=0}^{k} a_ib_i^{\rho} = 1$\\
\\
from problem 2 (see below for proof of this).\\
\\
We can calculate $\rho$\\
\\
$\rho = \sum_{i=1}^{n} a_ib_i^{\rho} = (\frac{1}{4})^1 + (\frac{3}{4})^1 = 1$\\
\\
Thus\\
\\
$T(x) = x(1 + \int_{1}^{x} \frac{g(u)}{u^{1 + 1}}du) = x(1 + \int_{1}^{x} \frac{1}{u}du)$\\
\\
$=xlnx$\\
\\
The runtime for this algorithm is loglinear. 


\medskip
\noindent{\bf Self-evaluation: 3 points}. We believe that the above is a complete and correct proof.
\end{solution}

\begin{solution}{2}\\\\

$T(x) = x^{\rho}(1 + \int_{1}^{x} \frac{g(u)}{u^{\rho + 1}}du) \, ,where \, \rho = \sum_{i=0}^{k} a_ib_i^{\rho} = 1$\\
The proof of the above statement is as follows:\\
\\
First we show that $g(x) = x^{\rho}\int_{b_ix}^{x} \frac{g(u)}{u^{\rho + 1}}du$\\\\
We will show this through induction:\\
\\
Assume $g(x) \geq c_1g(x)\,\,cg(x) \leq g(\lambda x) \leq c^{'}g(x)$ or in other words, g is scale free.\\
\\
$x^{\rho}\int_{b_ix}^{x} \frac{g(u)}{u^{\rho + 1}}du \geq x^{\rho}(x-b_i x) \frac{cg(x)}{max(x^{\rho + 1}, (b_i x)^{\rho + 1})}$\\
\\
$=\frac{(1-b_i)cg(x)}{max(1,b_i^{\rho + 1})} \geq c_1g(x)$\\
\\
This will hold as long as $c_1 \leq \frac{(1-b_i)c}{max(1,b_i^{\rho + 1})}$\\
\\
Now assuming $g(x) \leq c_2g(x)$, and knowing $cg(x) \leq g(\lambda x) \leq c^{'}g(x)$\\
\\
$x^{\rho}\int_{b_ix}^{x} \frac{g(u)}{u^{\rho + 1}}du \leq x^{\rho}(x-b_i x) \frac{c^{'}g(x)}{min(x^{\rho + 1}, (b_i x)^{\rho + 1})}
=\frac{(1-b_i)c^{'}g(x)}{min(1,b_i^{\rho + 1})}$\\
\\
$\leq c_2g(x)$\\
\\
This will hold as long as $c_2 \geq \frac{(1-b_i)c^{'}}{min(1,b_i^{\rho + 1})}$\\
\\
Now we can prove that $T(x) = x^{\rho}(1 + \int_{1}^{x} \frac{g(u)}{u^{\rho + 1}}du) \, ,where \, \rho = \sum_{i=0}^{k} a_ib_i^{\rho} = 1$\\
\\
First assume $T(x) \geq c_3T(x)$\\
\\
$T(x) = \sum_{i=1}^{k} a_i T(b_i x) + g(x)$\\
\\
$\geq \sum_{i=1}^{k} a_i c_3(b_i x)^{\rho}(1 + \int_{b_ix}^{x} \frac{g(u)}{u^{\rho + 1}}du) + g(x)$\\
\\
$=\sum_{i=1}^{k} a_i c_3(b_i x)^{\rho}(1 + \int_{1}^{x} \frac{g(u)}{u^{\rho + 1}}du - \int_{b_ix}^{x} \frac{g(u)}{u^{\rho + 1}}du) + g(x)$\\
\\
$=\sum_{i=1}^{k} a_i c_3(b_i x)^{\rho}(1 + \int_{1}^{x} \frac{g(u)}{u^{\rho + 1}}du - \sum_{i=1}^{k}a_i c_3 (b_i x)^{\rho} \int_{b_ix}^{x} \frac{g(u)}{u^{\rho + 1}}du) + g(x)$\\
\\
$=c_3 x^{\rho}(1 + \int_{1}^{x} \frac{g(u)}{u^{\rho + 1}}du) - (\sum_{i=1}^{k} a_i c_3 b_i^{\rho} x^{\rho} \int_{b_ix}^{x} \frac{g(u)}{u^{\rho + 1}}du) + g(x)$\\
\\
$\geq c_3 x^{\rho}(1 + \int_{1}^{x} \frac{g(u)}{u^{\rho + 1}}du) - c_3\sum_{i=1}^{k} a_i b_i^{\rho} c_2 g(x) + g(x)$\\
\\
$=c_3 x^{\rho}(1 + \int_{1}^{x} \frac{g(u)}{u^{\rho + 1}}du) - (c_2 c_3 - 1) * g(x)$\\
\\
$\geq c_3 x^{\rho}(1 + \int_{1}^{x} \frac{g(u)}{u^{\rho + 1}}du)$\\
\\
$=c_3*T(x)$\\
\\
This will hold as long as g(x) is nonnegative.\\
\\
Next, we will assume $T(x) \leq c_4 T(x)$ and use the same logic as we just did above.\\
\\
$T(x) = \sum_{i=1}^{k} a_i T(b_i x) + g(x)$\\
\\
$\leq \sum_{i=1}^{k} a_i c_4(b_i x)^{\rho}(1 + \int_{b_ix}^{x} \frac{g(u)}{u^{\rho + 1}}du) + g(x)$\\
\\
$=\sum_{i=1}^{k} a_i c_4(b_i x)^{\rho}(1 + \int_{1}^{x} \frac{g(u)}{u^{\rho + 1}}du - \int_{b_ix}^{x} \frac{g(u)}{u^{\rho + 1}}du) + g(x)$\\
\\
$=\sum_{i=1}^{k} a_i c_4(b_i x)^{\rho}(1 + \int_{1}^{x} \frac{g(u)}{u^{\rho + 1}}du - \sum_{i=1}^{k}a_i c_4 (b_i x)^{\rho} \int_{b_ix}^{x} \frac{g(u)}{u^{\rho + 1}}du) + g(x)$\\
\\
$=c_4 x^{\rho}(1 + \int_{1}^{x} \frac{g(u)}{u^{\rho + 1}}du) - (\sum_{i=1}^{k} a_i c_4 b_i^{\rho} x^{\rho} \int_{b_ix}^{x} \frac{g(u)}{u^{\rho + 1}}du) + g(x)$\\
\\
$\leq c_4 x^{\rho}(1 + \int_{1}^{x} \frac{g(u)}{u^{\rho + 1}}du) - c_4\sum_{i=1}^{k} a_i b_i^{\rho} c_1 g(x) + g(x)$\\
\\
$=c_4 x^{\rho}(1 + \int_{1}^{x} \frac{g(u)}{u^{\rho + 1}}du) - (c_1 c_4 - 1) * g(x)$\\
\\
$\leq c_4 x^{\rho}(1 + \int_{1}^{x} \frac{g(u)}{u^{\rho + 1}}du)$\\
\\
$=c_4*T(x)$\\
\\
Using the conclusions from these two steps, we have proved that\\
$T(x) = x^{\rho}(1 + \int_{1}^{x} \frac{g(u)}{u^{\rho + 1}}du) \, ,where \, \rho = \sum_{i=0}^{k} a_ib_i^{\rho} = 1$\\


\medskip
\noindent{\bf Self-evaluation: 3 point}. We believe that the above is a complete and correct proof.

\end{solution}

\begin{solution}{3}\\\\
Because we need to achieve $O(logn)$ runtime we have to use a modified binary search that narrows down
our search space in one array at a time. The pseudocode for the algorithm is described below:\\

\noindent$find(x, y, i)$\\
\indent  $if\,x.length == 0$\\
\indent\indent    $return y[i]$\\
\indent  $if\,y.length == 0$\\
\indent\indent    $return x[i]$\\
\indent  $if\,i == 1$\\
\indent\indent   $return\,min(x[i], y[i])$\\
\indent  $a = min(\frac{1}{2}i, x.length)$\\
\indent $b = min(\frac{1}{2}i, y.length)$\\
\indent$if\,x[a] \leq y[b]$\\
\indent\indent  $find(x[a+1:end],y,i-a)$\\
\indent$else$\\
\indent\indent  $find(x,y[b+1:end],i-b)$\\

\medskip
\noindent{\bf Self-evaluation: 3 point}. We believe the algorithm we have is correct in the context of the problem.
\end{solution}

\end{document}
